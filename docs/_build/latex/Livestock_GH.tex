%% Generated by Sphinx.
\def\sphinxdocclass{report}
\documentclass[letterpaper,10pt,english]{sphinxmanual}
\ifdefined\pdfpxdimen
   \let\sphinxpxdimen\pdfpxdimen\else\newdimen\sphinxpxdimen
\fi \sphinxpxdimen=.75bp\relax

\usepackage[utf8]{inputenc}
\ifdefined\DeclareUnicodeCharacter
 \ifdefined\DeclareUnicodeCharacterAsOptional
  \DeclareUnicodeCharacter{"00A0}{\nobreakspace}
  \DeclareUnicodeCharacter{"2500}{\sphinxunichar{2500}}
  \DeclareUnicodeCharacter{"2502}{\sphinxunichar{2502}}
  \DeclareUnicodeCharacter{"2514}{\sphinxunichar{2514}}
  \DeclareUnicodeCharacter{"251C}{\sphinxunichar{251C}}
  \DeclareUnicodeCharacter{"2572}{\textbackslash}
 \else
  \DeclareUnicodeCharacter{00A0}{\nobreakspace}
  \DeclareUnicodeCharacter{2500}{\sphinxunichar{2500}}
  \DeclareUnicodeCharacter{2502}{\sphinxunichar{2502}}
  \DeclareUnicodeCharacter{2514}{\sphinxunichar{2514}}
  \DeclareUnicodeCharacter{251C}{\sphinxunichar{251C}}
  \DeclareUnicodeCharacter{2572}{\textbackslash}
 \fi
\fi
\usepackage{cmap}
\usepackage[T1]{fontenc}
\usepackage{amsmath,amssymb,amstext}
\usepackage{babel}
\usepackage{times}
\usepackage[Bjarne]{fncychap}
\usepackage[dontkeepoldnames]{sphinx}

\usepackage{geometry}

% Include hyperref last.
\usepackage{hyperref}
% Fix anchor placement for figures with captions.
\usepackage{hypcap}% it must be loaded after hyperref.
% Set up styles of URL: it should be placed after hyperref.
\urlstyle{same}
\addto\captionsenglish{\renewcommand{\contentsname}{Contents:}}

\addto\captionsenglish{\renewcommand{\figurename}{Fig.}}
\addto\captionsenglish{\renewcommand{\tablename}{Table}}
\addto\captionsenglish{\renewcommand{\literalblockname}{Listing}}

\addto\captionsenglish{\renewcommand{\literalblockcontinuedname}{continued from previous page}}
\addto\captionsenglish{\renewcommand{\literalblockcontinuesname}{continues on next page}}

\addto\extrasenglish{\def\pageautorefname{page}}

\setcounter{tocdepth}{1}



\title{Livestock\_GH Documentation}
\date{Jan 05, 2018}
\release{.0}
\author{Christian Kongsgaard Nielsen}
\newcommand{\sphinxlogo}{\vbox{}}
\renewcommand{\releasename}{Release}
\makeindex

\begin{document}

\maketitle
\sphinxtableofcontents
\phantomsection\label{\detokenize{index::doc}}


Livestock is the name of the library of components that has been developed for this thesis.
Livestock consists of a series of Grasshopper Python Script components and a underlying collection of Python scripts
and a PyPI \textendash{} Python Package Index - (Foundation 2017) package.

Each component is modelled as a placeholder with the Grasshopper Python Script for a Python class implementation of the
component. This makes it possible to create and edit the components without having Grasshopper open, thus making it
possible to use better Python editors than the build-in Grasshopper one \textendash{} in this case PyCharm (JetBrains 2017) was
used. Besides easing the workflow of this author, it also solves the problem of updating the components in the future.
Some Grasshopper plug-ins such as Ladybug Tools (Sadeghipour Roudsari and Pak 2013), have written most of the source
code directly in the Grasshopper Pythons Script components and every time an update is made or a bug is fixed the
component has to be replaced with a newer version, which is troublesome. By having the Grasshopper component as a
placeholder, this can be avoided. The only time a Livestock component should be replaced is when new inputs or outputs
are added to the component, which is far less often, than when bug fixes are made.

As stated above Livestock also have a PyPI package \textendash{} called livestock. At first this might sound a bit strange or as a
duplicate of the Python scripts imported into Grasshopper \textendash{} but it is not. The Grasshopper Python Script component runs
IronPython (Hugunin and Viehland 2008), which is an implementation of Python for the .NET framework. IronPython comes as
Python 2 and has long been undermaintained and developed. Standard Python or CPython is well maintained currently on
version 3.6. It goes without saying that CPython has undergone a lot development that IronPython has not and includes
a whole bunch of new features. Furthermore, are the packages on PyPI mainly targeting CPython and many cannot be used
in IronPython. This drastically decreases the usability the usability of the IronPython in the opinion of this author
and very specific CMF cannot be ran from IronPython, therefore a CPython package and scripts are made. The CPython
Livestock package \textendash{} from now on mentioned as the Livestock package \textendash{} is used when ever larger computations or CMF are
needed.


\chapter{Livestock Grasshopper Components}
\label{\detokenize{components::doc}}\label{\detokenize{components:livestock-grasshopper-components}}\label{\detokenize{components:welcome-to-livestock-gh-s-documentation}}

\section{0 \textbar{} Miscellaneous}
\label{\detokenize{components:miscellaneous}}
\sphinxstylestrong{Livestock Python Executor}
\begin{quote}\begin{description}
\item[{Description}] \leavevmode
\begin{DUlineblock}{0em}
\item[] Path to python executor.
\end{DUlineblock}

\item[{Inputs}] \leavevmode\begin{quote}\begin{description}
\item[{1.}] \leavevmode\begin{quote}\begin{description}
\item[{Name}] \leavevmode
PythonPath

\item[{Description}] \leavevmode
Path to python.exe

\item[{Data Access}] \leavevmode
Item

\item[{Default Value}] \leavevmode
\begin{DUlineblock}{0em}
\item[] None
\end{DUlineblock}

\end{description}\end{quote}

\end{description}\end{quote}

\item[{Outputs}] \leavevmode\begin{quote}\begin{description}
\item[{1.}] \leavevmode\begin{quote}\begin{description}
\item[{Name}] \leavevmode
readMe!

\item[{Description}] \leavevmode
\begin{DUlineblock}{0em}
\item[] In case of any errors, it will be shown here.
\end{DUlineblock}

\end{description}\end{quote}

\item[{2.}] \leavevmode\begin{quote}\begin{description}
\item[{Name}] \leavevmode
BoundaryCondition

\item[{Description}] \leavevmode
\begin{DUlineblock}{0em}
\item[] Livestock Boundary Conditions.
\end{DUlineblock}

\end{description}\end{quote}

\end{description}\end{quote}

\end{description}\end{quote}

\sphinxstylestrong{Livestock SSH Connection}
\begin{quote}\begin{description}
\item[{Description}] \leavevmode
\begin{DUlineblock}{0em}
\item[] Setup SSH connection.
\item[] Icon based on art from Arthur Shlain from the Noun Project.
\end{DUlineblock}

\item[{Inputs}] \leavevmode\begin{quote}\begin{description}
\item[{1.}] \leavevmode\begin{quote}\begin{description}
\item[{Name}] \leavevmode
IP

\item[{Description}] \leavevmode
IP Address for SSH connection.

\item[{Data Access}] \leavevmode
Item

\item[{Default Value}] \leavevmode
\begin{DUlineblock}{0em}
\item[] None
\end{DUlineblock}

\end{description}\end{quote}

\item[{2.}] \leavevmode\begin{quote}\begin{description}
\item[{Name}] \leavevmode
Port

\item[{Description}] \leavevmode
Port for SSH connection.

\item[{Data Access}] \leavevmode
Item

\item[{Default Value}] \leavevmode
\begin{DUlineblock}{0em}
\item[] None
\end{DUlineblock}

\end{description}\end{quote}

\item[{3.}] \leavevmode\begin{quote}\begin{description}
\item[{Name}] \leavevmode
Username

\item[{Description}] \leavevmode
Username for SSH connection.

\item[{Data Access}] \leavevmode
Item

\item[{Default Value}] \leavevmode
\begin{DUlineblock}{0em}
\item[] None
\end{DUlineblock}

\end{description}\end{quote}

\item[{4.}] \leavevmode\begin{quote}\begin{description}
\item[{Name}] \leavevmode
Password

\item[{Description}] \leavevmode
Password for SSH connection.

\item[{Data Access}] \leavevmode
Item

\item[{Default Value}] \leavevmode
\begin{DUlineblock}{0em}
\item[] None
\end{DUlineblock}

\end{description}\end{quote}

\end{description}\end{quote}

\item[{Outputs}] \leavevmode\begin{quote}\begin{description}
\item[{1.}] \leavevmode\begin{quote}\begin{description}
\item[{Name}] \leavevmode
readMe!

\item[{Description}] \leavevmode
\begin{DUlineblock}{0em}
\item[] In case of any errors, it will be shown here.
\end{DUlineblock}

\end{description}\end{quote}

\end{description}\end{quote}

\end{description}\end{quote}


\section{1 \textbar{} Geometry}
\label{\detokenize{components:geometry}}
\sphinxstylestrong{Livestock Load Mesh}
\begin{quote}\begin{description}
\item[{Description}] \leavevmode
Loads a mesh.

\item[{Inputs}] \leavevmode\begin{quote}\begin{description}
\item[{1.}] \leavevmode\begin{quote}\begin{description}
\item[{Name}] \leavevmode
Filename

\item[{Description}] \leavevmode
Directory and file name of mesh.

\item[{Data Access}] \leavevmode
Item

\item[{Default Value}] \leavevmode
\begin{DUlineblock}{0em}
\item[] None
\end{DUlineblock}

\end{description}\end{quote}

\item[{2.}] \leavevmode\begin{quote}\begin{description}
\item[{Name}] \leavevmode
Load

\item[{Description}] \leavevmode
Activates the component.

\item[{Data Access}] \leavevmode
Item

\item[{Default Value}] \leavevmode
\begin{DUlineblock}{0em}
\item[] False
\end{DUlineblock}

\end{description}\end{quote}

\end{description}\end{quote}

\item[{Outputs}] \leavevmode\begin{quote}\begin{description}
\item[{1.}] \leavevmode\begin{quote}\begin{description}
\item[{Name}] \leavevmode
readMe!

\item[{Description}] \leavevmode
\begin{DUlineblock}{0em}
\item[] In case of any errors, it will be shown here.
\end{DUlineblock}

\end{description}\end{quote}

\item[{2.}] \leavevmode\begin{quote}\begin{description}
\item[{Name}] \leavevmode
Mesh

\item[{Description}] \leavevmode
\begin{DUlineblock}{0em}
\item[] Loaded mesh.
\end{DUlineblock}

\end{description}\end{quote}

\item[{3.}] \leavevmode\begin{quote}\begin{description}
\item[{Name}] \leavevmode
MeshData

\item[{Description}] \leavevmode
\begin{DUlineblock}{0em}
\item[] Additional data if any.
\end{DUlineblock}

\end{description}\end{quote}

\end{description}\end{quote}

\end{description}\end{quote}

\sphinxstylestrong{Livestock Save Mesh}
\begin{quote}\begin{description}
\item[{Description}] \leavevmode
Saves a mesh and additional data

\item[{Inputs}] \leavevmode\begin{quote}\begin{description}
\item[{1.}] \leavevmode\begin{quote}\begin{description}
\item[{Name}] \leavevmode
Mesh

\item[{Description}] \leavevmode
Mesh to save.

\item[{Data Access}] \leavevmode
Item

\item[{Default Value}] \leavevmode
\begin{DUlineblock}{0em}
\item[] None
\end{DUlineblock}

\end{description}\end{quote}

\item[{2.}] \leavevmode\begin{quote}\begin{description}
\item[{Name}] \leavevmode
Data

\item[{Description}] \leavevmode
Additional data if any.

\item[{Data Access}] \leavevmode
Item

\item[{Default Value}] \leavevmode
\begin{DUlineblock}{0em}
\item[] None
\end{DUlineblock}

\end{description}\end{quote}

\item[{3.}] \leavevmode\begin{quote}\begin{description}
\item[{Name}] \leavevmode
Directory

\item[{Description}] \leavevmode
File path to save mesh to.

\item[{Data Access}] \leavevmode
Item

\item[{Default Value}] \leavevmode
\begin{DUlineblock}{0em}
\item[] None
\end{DUlineblock}

\end{description}\end{quote}

\item[{4.}] \leavevmode\begin{quote}\begin{description}
\item[{Name}] \leavevmode
Filename

\item[{Description}] \leavevmode
File name.

\item[{Data Access}] \leavevmode
Item

\item[{Default Value}] \leavevmode
\begin{DUlineblock}{0em}
\item[] None
\end{DUlineblock}

\end{description}\end{quote}

\item[{5.}] \leavevmode\begin{quote}\begin{description}
\item[{Name}] \leavevmode
Save

\item[{Description}] \leavevmode
Activates the component.

\item[{Data Access}] \leavevmode
Item

\item[{Default Value}] \leavevmode
\begin{DUlineblock}{0em}
\item[] False
\end{DUlineblock}

\end{description}\end{quote}

\end{description}\end{quote}

\item[{Outputs}] \leavevmode\begin{quote}\begin{description}
\item[{1.}] \leavevmode\begin{quote}\begin{description}
\item[{Name}] \leavevmode
readMe!

\item[{Description}] \leavevmode
\begin{DUlineblock}{0em}
\item[] In case of any errors, it will be shown here.
\end{DUlineblock}

\end{description}\end{quote}

\end{description}\end{quote}

\end{description}\end{quote}


\section{3 \textbar{} CMF}
\label{\detokenize{components:cmf}}
\sphinxstylestrong{Livestock CMF Ground}
\begin{quote}\begin{description}
\item[{Description}] \leavevmode
\begin{DUlineblock}{0em}
\item[] Generates CMF ground.
\item[] Icon art based created by Ben Davis from the Noun Project.
\end{DUlineblock}

\item[{Inputs}] \leavevmode\begin{quote}\begin{description}
\item[{1.}] \leavevmode\begin{quote}\begin{description}
\item[{Name}] \leavevmode
Layers

\item[{Description}] \leavevmode
Soil layers to add to the mesh in m.

\item[{Data Access}] \leavevmode
Item

\item[{Default Value}] \leavevmode
\begin{DUlineblock}{0em}
\item[] 0
\end{DUlineblock}

\end{description}\end{quote}

\item[{2.}] \leavevmode\begin{quote}\begin{description}
\item[{Name}] \leavevmode
RetentionCurve

\item[{Description}] \leavevmode
Livestock CMF Retention Curve.

\item[{Data Access}] \leavevmode
Item

\item[{Default Value}] \leavevmode
\begin{DUlineblock}{0em}
\item[] None
\end{DUlineblock}

\end{description}\end{quote}

\item[{3.}] \leavevmode\begin{quote}\begin{description}
\item[{Name}] \leavevmode
VegetationProperties

\item[{Description}] \leavevmode
Input from Livestock CMF Vegetation Properties.

\item[{Data Access}] \leavevmode
Item

\item[{Default Value}] \leavevmode
\begin{DUlineblock}{0em}
\item[] None
\end{DUlineblock}

\end{description}\end{quote}

\item[{4.}] \leavevmode\begin{quote}\begin{description}
\item[{Name}] \leavevmode
SaturatedDepth

\item[{Description}] \leavevmode
Initial saturated depth in m. It is depth where the groundwater is located. Default is set
to 3m.

\item[{Data Access}] \leavevmode
Item

\item[{Default Value}] \leavevmode
\begin{DUlineblock}{0em}
\item[] 3
\end{DUlineblock}

\end{description}\end{quote}

\item[{5.}] \leavevmode\begin{quote}\begin{description}
\item[{Name}] \leavevmode
FaceIndices

\item[{Description}] \leavevmode
List of face indices, on where the ground properties are applied.

\item[{Data Access}] \leavevmode
List

\item[{Default Value}] \leavevmode
\begin{DUlineblock}{0em}
\item[] None
\end{DUlineblock}

\end{description}\end{quote}

\item[{6.}] \leavevmode\begin{quote}\begin{description}
\item[{Name}] \leavevmode
ETMethod

\item[{Description}] \leavevmode
\begin{DUlineblock}{0em}
\item[] Set method to calculate evapotranspiration.
\item[] 0: No evapotranspiration.
\item[] 1: Penman-Monteith.
\item[] 2: Shuttleworth-Wallace.
\item[] Default is set to Shuttleworth-Wallace.
\end{DUlineblock}

\item[{Data Access}] \leavevmode
Item

\item[{Default Value}] \leavevmode
\begin{DUlineblock}{0em}
\item[] 2
\end{DUlineblock}

\end{description}\end{quote}

\item[{7.}] \leavevmode\begin{quote}\begin{description}
\item[{Name}] \leavevmode
Manning

\item[{Description}] \leavevmode
Set Manning roughness. If not set CMF calculates it from the above given values.

\item[{Data Access}] \leavevmode
Item

\item[{Default Value}] \leavevmode
\begin{DUlineblock}{0em}
\item[] None
\end{DUlineblock}

\end{description}\end{quote}

\item[{8.}] \leavevmode\begin{quote}\begin{description}
\item[{Name}] \leavevmode
PuddleDepth

\item[{Description}] \leavevmode
Set puddle depth. Puddle depth is the height were run-off begins.

\item[{Data Access}] \leavevmode
Item

\item[{Default Value}] \leavevmode
\begin{DUlineblock}{0em}
\item[] 0.01
\end{DUlineblock}

\end{description}\end{quote}

\end{description}\end{quote}

\item[{Outputs}] \leavevmode\begin{quote}\begin{description}
\item[{1.}] \leavevmode\begin{quote}\begin{description}
\item[{Name}] \leavevmode
readMe!

\item[{Description}] \leavevmode
In case of any errors, it will be shown here.

\end{description}\end{quote}

\item[{2.}] \leavevmode\begin{quote}\begin{description}
\item[{Name}] \leavevmode
Ground

\item[{Description}] \leavevmode
Livestock Ground Data Class.

\end{description}\end{quote}

\end{description}\end{quote}

\end{description}\end{quote}

\sphinxstylestrong{Livestock CMF Weather}
\begin{quote}\begin{description}
\item[{Description}] \leavevmode
\begin{DUlineblock}{0em}
\item[] Generates CMF weather.
\item[] Icon art based created by Adrien Coquet from the Noun Project.
\end{DUlineblock}

\item[{Inputs}] \leavevmode\begin{quote}\begin{description}
\item[{1.}] \leavevmode\begin{quote}\begin{description}
\item[{Name}] \leavevmode
Temperature

\item[{Description}] \leavevmode
Temperature in C. Either a list or a tree where the number of branches is equal to the number
of mesh faces.

\item[{Data Access}] \leavevmode
Tree

\item[{Default Value}] \leavevmode
\begin{DUlineblock}{0em}
\item[] None
\end{DUlineblock}

\end{description}\end{quote}

\item[{2.}] \leavevmode\begin{quote}\begin{description}
\item[{Name}] \leavevmode
WindSpeed

\item[{Description}] \leavevmode
Wind speed in m/s. Either a list or a tree where the number of branches is equal to the number
of mesh faces.

\item[{Data Access}] \leavevmode
Tree

\item[{Default Value}] \leavevmode
\begin{DUlineblock}{0em}
\item[] None
\end{DUlineblock}

\end{description}\end{quote}

\item[{3.}] \leavevmode\begin{quote}\begin{description}
\item[{Name}] \leavevmode
RelativeHumidity

\item[{Description}] \leavevmode
Relative humidity in \%. Either a list or a tree where the number of branches is equal to the number
of mesh faces.

\item[{Data Access}] \leavevmode
Tree

\item[{Default Value}] \leavevmode
\begin{DUlineblock}{0em}
\item[] None
\end{DUlineblock}

\end{description}\end{quote}

\item[{4.}] \leavevmode\begin{quote}\begin{description}
\item[{Name}] \leavevmode
CloudCover

\item[{Description}] \leavevmode
Cloud cover, unitless between 0 and 1. Either a list or a tree where the number of branches is equal to the number
of mesh faces.

\item[{Data Access}] \leavevmode
Tree

\item[{Default Value}] \leavevmode
\begin{DUlineblock}{0em}
\item[] None
\end{DUlineblock}

\end{description}\end{quote}

\item[{5.}] \leavevmode\begin{quote}\begin{description}
\item[{Name}] \leavevmode
GlobalRadiation

\item[{Description}] \leavevmode
Global Radiation in W/m:sup:\sphinxtitleref{2}. Either a list or a tree where the number of branches is equal to the number
of mesh faces.

\item[{Data Access}] \leavevmode
Tree

\item[{Default Value}] \leavevmode
\begin{DUlineblock}{0em}
\item[] None
\end{DUlineblock}

\end{description}\end{quote}

\item[{6.}] \leavevmode\begin{quote}\begin{description}
\item[{Name}] \leavevmode
Rain

\item[{Description}] \leavevmode
Horizontal precipitation in mm/h. Either a list or a tree where the number of branches is equal to the number
of mesh faces.

\item[{Data Access}] \leavevmode
Tree

\item[{Default Value}] \leavevmode
\begin{DUlineblock}{0em}
\item[] None
\end{DUlineblock}

\end{description}\end{quote}

\item[{7.}] \leavevmode\begin{quote}\begin{description}
\item[{Name}] \leavevmode
GroundTemperature

\item[{Description}] \leavevmode
Ground temperature in C. Either a list or a tree where the number of branches is equal to the number
of mesh faces.

\item[{Data Access}] \leavevmode
Tree

\item[{Default Value}] \leavevmode
\begin{DUlineblock}{0em}
\item[] None
\end{DUlineblock}

\end{description}\end{quote}

\item[{8.}] \leavevmode\begin{quote}\begin{description}
\item[{Name}] \leavevmode
Location

\item[{Description}] \leavevmode
A Ladybug Tools Locations.

\item[{Data Access}] \leavevmode
Item

\item[{Default Value}] \leavevmode
\begin{DUlineblock}{0em}
\item[] None
\end{DUlineblock}

\end{description}\end{quote}

\item[{9.}] \leavevmode\begin{quote}\begin{description}
\item[{Name}] \leavevmode
MeshFaceCount

\item[{Description}] \leavevmode
Number of faces in the ground mesh.

\item[{Data Access}] \leavevmode
Item

\item[{Default Value}] \leavevmode
\begin{DUlineblock}{0em}
\item[] None
\end{DUlineblock}

\end{description}\end{quote}

\end{description}\end{quote}

\item[{Outputs}] \leavevmode\begin{quote}\begin{description}
\item[{1.}] \leavevmode\begin{quote}\begin{description}
\item[{Name}] \leavevmode
readMe!

\item[{Description}] \leavevmode
\begin{DUlineblock}{0em}
\item[] In case of any errors, it will be shown here.
\end{DUlineblock}

\end{description}\end{quote}

\item[{2.}] \leavevmode\begin{quote}\begin{description}
\item[{Name}] \leavevmode
Weather

\item[{Description}] \leavevmode
\begin{DUlineblock}{0em}
\item[] Livestock Weather Data Class.
\end{DUlineblock}

\end{description}\end{quote}

\end{description}\end{quote}

\end{description}\end{quote}

\sphinxstylestrong{Livestock CMF Vegetation Properties}
\begin{quote}\begin{description}
\item[{Description}] \leavevmode
\begin{DUlineblock}{0em}
\item[] Generates CMF Vegetation Properties
\item[] Icon art based created by Ben Davis from the Noun Project.
\end{DUlineblock}

\item[{Inputs}] \leavevmode\begin{quote}\begin{description}
\item[{1.}] \leavevmode\begin{quote}\begin{description}
\item[{Name}] \leavevmode
Property

\item[{Description}] \leavevmode
0-1 grasses. 2-6 soils. Default is set to 0

\item[{Data Access}] \leavevmode
Item

\item[{Default Value}] \leavevmode
\begin{DUlineblock}{0em}
\item[] 0
\end{DUlineblock}

\end{description}\end{quote}

\end{description}\end{quote}

\item[{Outputs}] \leavevmode\begin{quote}\begin{description}
\item[{1.}] \leavevmode\begin{quote}\begin{description}
\item[{Name}] \leavevmode
readMe!

\item[{Description}] \leavevmode
\begin{DUlineblock}{0em}
\item[] In case of any errors, it will be shown here.
\end{DUlineblock}

\end{description}\end{quote}

\item[{2.}] \leavevmode\begin{quote}\begin{description}
\item[{Name}] \leavevmode
Units

\item[{Description}] \leavevmode
\begin{DUlineblock}{0em}
\item[] Shows the units of the surface values.
\end{DUlineblock}

\end{description}\end{quote}

\item[{3.}] \leavevmode\begin{quote}\begin{description}
\item[{Name}] \leavevmode
VegetationValues

\item[{Description}] \leavevmode
\begin{DUlineblock}{0em}
\item[] Chosen vegetation property values.
\end{DUlineblock}

\end{description}\end{quote}

\item[{4.}] \leavevmode\begin{quote}\begin{description}
\item[{Name}] \leavevmode
VegetationProperties

\item[{Description}] \leavevmode
\begin{DUlineblock}{0em}
\item[] Livestock Vegetation Property Data.
\end{DUlineblock}

\end{description}\end{quote}

\end{description}\end{quote}

\end{description}\end{quote}

\sphinxstylestrong{Livestock CMF Synthetic Tree}
\begin{quote}\begin{description}
\item[{Description}] \leavevmode
\begin{DUlineblock}{0em}
\item[] Generates a synthetic tree
\end{DUlineblock}

\item[{Inputs}] \leavevmode\begin{quote}\begin{description}
\item[{1.}] \leavevmode\begin{quote}\begin{description}
\item[{Name}] \leavevmode
FaceIndex

\item[{Description}] \leavevmode
Mesh face index where tree is placed

\item[{Data Access}] \leavevmode
Item

\item[{Default Value}] \leavevmode
\begin{DUlineblock}{0em}
\item[] None
\end{DUlineblock}

\end{description}\end{quote}

\item[{2.}] \leavevmode\begin{quote}\begin{description}
\item[{Name}] \leavevmode
TreeType

\item[{Description}] \leavevmode
Tree types: 0 - Deciduous, 1 - Coniferous, 2 - Shrubs. Default is deciduous.

\item[{Data Access}] \leavevmode
Item

\item[{Default Value}] \leavevmode
\begin{DUlineblock}{0em}
\item[] 0
\end{DUlineblock}

\end{description}\end{quote}

\item[{3.}] \leavevmode\begin{quote}\begin{description}
\item[{Name}] \leavevmode
Height

\item[{Description}] \leavevmode
Height of tree in meters. Default is set to 10m

\item[{Data Access}] \leavevmode
Item

\item[{Default Value}] \leavevmode
\begin{DUlineblock}{0em}
\item[] 10
\end{DUlineblock}

\end{description}\end{quote}

\end{description}\end{quote}

\item[{Outputs}] \leavevmode\begin{quote}\begin{description}
\item[{1.}] \leavevmode\begin{quote}\begin{description}
\item[{Name}] \leavevmode
readMe!

\item[{Description}] \leavevmode
\begin{DUlineblock}{0em}
\item[] In case of any errors, it will be shown here.
\end{DUlineblock}

\end{description}\end{quote}

\item[{2.}] \leavevmode\begin{quote}\begin{description}
\item[{Name}] \leavevmode
Units

\item[{Description}] \leavevmode
\begin{DUlineblock}{0em}
\item[] Shows the units of the tree values.
\end{DUlineblock}

\end{description}\end{quote}

\item[{3.}] \leavevmode\begin{quote}\begin{description}
\item[{Name}] \leavevmode
TreeValues

\item[{Description}] \leavevmode
\begin{DUlineblock}{0em}
\item[] Chosen tree properties values.
\end{DUlineblock}

\end{description}\end{quote}

\item[{4.}] \leavevmode\begin{quote}\begin{description}
\item[{Name}] \leavevmode
TreeProperties

\item[{Description}] \leavevmode
\begin{DUlineblock}{0em}
\item[] Livestock tree properties data.
\end{DUlineblock}

\end{description}\end{quote}

\end{description}\end{quote}

\end{description}\end{quote}

\sphinxstylestrong{Livestock CMF Retention Curve}
\begin{quote}\begin{description}
\item[{Description}] \leavevmode
Generates a retention curve.

\item[{Inputs}] \leavevmode\begin{quote}\begin{description}
\item[{1.}] \leavevmode\begin{quote}\begin{description}
\item[{Name}] \leavevmode
SoilIndex

\item[{Description}] \leavevmode
Index for choosing soil type. Index from 0-5. Default is set to 0, which is the default CMF
retention curve.

\item[{Data Access}] \leavevmode
Item

\item[{Default Value}] \leavevmode
\begin{DUlineblock}{0em}
\item[] 0
\end{DUlineblock}

\end{description}\end{quote}

\item[{2.}] \leavevmode\begin{quote}\begin{description}
\item[{Name}] \leavevmode
K\_sat

\item[{Description}] \leavevmode
Saturated conductivity in m/day.

\item[{Data Access}] \leavevmode
Item

\item[{Default Value}] \leavevmode
\begin{DUlineblock}{0em}
\item[] None
\end{DUlineblock}

\end{description}\end{quote}

\item[{3.}] \leavevmode\begin{quote}\begin{description}
\item[{Name}] \leavevmode
Phi

\item[{Description}] \leavevmode
Porosity in m3/m3.

\item[{Data Access}] \leavevmode
Item

\item[{Default Value}] \leavevmode
\begin{DUlineblock}{0em}
\item[] None
\end{DUlineblock}

\end{description}\end{quote}

\item[{4.}] \leavevmode\begin{quote}\begin{description}
\item[{Name}] \leavevmode
Alpha

\item[{Description}] \leavevmode
Inverse of water entry potential in 1/cm.

\item[{Data Access}] \leavevmode
Item

\item[{Default Value}] \leavevmode
\begin{DUlineblock}{0em}
\item[] 0
\end{DUlineblock}

\end{description}\end{quote}

\item[{5.}] \leavevmode\begin{quote}\begin{description}
\item[{Name}] \leavevmode
N

\item[{Description}] \leavevmode
Pore size distribution parameter is unitless.

\item[{Data Access}] \leavevmode
Item

\item[{Default Value}] \leavevmode
\begin{DUlineblock}{0em}
\item[] None
\end{DUlineblock}

\end{description}\end{quote}

\item[{6.}] \leavevmode\begin{quote}\begin{description}
\item[{Name}] \leavevmode
M

\item[{Description}] \leavevmode
VanGenuchten m (if negative, 1-1/n is used) is unitless.

\item[{Data Access}] \leavevmode
Item

\item[{Default Value}] \leavevmode
\begin{DUlineblock}{0em}
\item[] None
\end{DUlineblock}

\end{description}\end{quote}

\item[{6.}] \leavevmode\begin{quote}\begin{description}
\item[{Name}] \leavevmode
L

\item[{Description}] \leavevmode
Mualem tortoisivity is unitless.

\item[{Data Access}] \leavevmode
Item

\item[{Default Value}] \leavevmode
\begin{DUlineblock}{0em}
\item[] None
\end{DUlineblock}

\end{description}\end{quote}

\end{description}\end{quote}

\item[{Outputs}] \leavevmode\begin{quote}\begin{description}
\item[{1.}] \leavevmode\begin{quote}\begin{description}
\item[{Name}] \leavevmode
readMe!

\item[{Description}] \leavevmode
\begin{DUlineblock}{0em}
\item[] In case of any errors, it will be shown here.
\end{DUlineblock}

\end{description}\end{quote}

\item[{2.}] \leavevmode\begin{quote}\begin{description}
\item[{Name}] \leavevmode
Units

\item[{Description}] \leavevmode
\begin{DUlineblock}{0em}
\item[] Shows the units of the curve values.
\end{DUlineblock}

\end{description}\end{quote}

\item[{3.}] \leavevmode\begin{quote}\begin{description}
\item[{Name}] \leavevmode
CurveValues

\item[{Description}] \leavevmode
\begin{DUlineblock}{0em}
\item[] Chosen curve properties values.
\end{DUlineblock}

\end{description}\end{quote}

\item[{4.}] \leavevmode\begin{quote}\begin{description}
\item[{Name}] \leavevmode
RetentionCurve

\item[{Description}] \leavevmode
\begin{DUlineblock}{0em}
\item[] Livestock Retention Curve.
\end{DUlineblock}

\end{description}\end{quote}

\end{description}\end{quote}

\end{description}\end{quote}

\sphinxstylestrong{Livestock CMF Solve}
\begin{quote}\begin{description}
\item[{Description}] \leavevmode
\begin{DUlineblock}{0em}
\item[] Solves CMF Case.
\item[] Icon art based on Vectors Market from the Noun Project.
\end{DUlineblock}

\item[{Inputs}] \leavevmode\begin{quote}\begin{description}
\item[{1.}] \leavevmode\begin{quote}\begin{description}
\item[{Name}] \leavevmode
Mesh

\item[{Description}] \leavevmode
Topography as a mesh.

\item[{Data Access}] \leavevmode
Item

\item[{Default Value}] \leavevmode
\begin{DUlineblock}{0em}
\item[] None
\end{DUlineblock}

\end{description}\end{quote}

\item[{2.}] \leavevmode\begin{quote}\begin{description}
\item[{Name}] \leavevmode
Ground

\item[{Description}] \leavevmode
Input from Livestock CMF Ground.

\item[{Data Access}] \leavevmode
List

\item[{Default Value}] \leavevmode
\begin{DUlineblock}{0em}
\item[] None
\end{DUlineblock}

\end{description}\end{quote}

\item[{3.}] \leavevmode\begin{quote}\begin{description}
\item[{Name}] \leavevmode
Weather

\item[{Description}] \leavevmode
Input from Livestock CMF Weather.

\item[{Data Access}] \leavevmode
Item

\item[{Default Value}] \leavevmode
\begin{DUlineblock}{0em}
\item[] None
\end{DUlineblock}

\end{description}\end{quote}

\item[{4.}] \leavevmode\begin{quote}\begin{description}
\item[{Name}] \leavevmode
Trees

\item[{Description}] \leavevmode
Input from Livestock CMF Tree.

\item[{Data Access}] \leavevmode
List

\item[{Default Value}] \leavevmode
\begin{DUlineblock}{0em}
\item[] None
\end{DUlineblock}

\end{description}\end{quote}

\item[{5.}] \leavevmode\begin{quote}\begin{description}
\item[{Name}] \leavevmode
Stream

\item[{Description}] \leavevmode
Input from Livestock CMF Stream. \sphinxstylestrong{Currently not working.}

\item[{Data Access}] \leavevmode
Item

\item[{Default Value}] \leavevmode
\begin{DUlineblock}{0em}
\item[] None
\end{DUlineblock}

\end{description}\end{quote}

\item[{6.}] \leavevmode\begin{quote}\begin{description}
\item[{Name}] \leavevmode
BoundaryConditions

\item[{Description}] \leavevmode
Input from Livestock CMF Boundary Condition.

\item[{Data Access}] \leavevmode
List

\item[{Default Value}] \leavevmode
\begin{DUlineblock}{0em}
\item[] None
\end{DUlineblock}

\end{description}\end{quote}

\item[{7.}] \leavevmode\begin{quote}\begin{description}
\item[{Name}] \leavevmode
SolverSettings

\item[{Description}] \leavevmode
Input from Livestock CMF Solver Settings.

\item[{Data Access}] \leavevmode
Item

\item[{Default Value}] \leavevmode
\begin{DUlineblock}{0em}
\item[] None
\end{DUlineblock}

\end{description}\end{quote}

\item[{8.}] \leavevmode\begin{quote}\begin{description}
\item[{Name}] \leavevmode
Folder

\item[{Description}] \leavevmode
Path to folder. Default is Desktop.

\item[{Data Access}] \leavevmode
Item

\item[{Default Value}] \leavevmode
\begin{DUlineblock}{0em}
\item[] os.path.join(os.environ{[}“HOMEPATH”{]}, “Desktop”)\}
\end{DUlineblock}

\end{description}\end{quote}

\item[{9.}] \leavevmode\begin{quote}\begin{description}
\item[{Name}] \leavevmode
CaseName

\item[{Description}] \leavevmode
Case name as string. Default is CMF

\item[{Data Access}] \leavevmode
Item

\item[{Default Value}] \leavevmode
\begin{DUlineblock}{0em}
\item[] CMF
\end{DUlineblock}

\end{description}\end{quote}

\item[{10.}] \leavevmode\begin{quote}\begin{description}
\item[{Name}] \leavevmode
Outputs

\item[{Description}] \leavevmode
Connect Livestock Outputs.

\item[{Data Access}] \leavevmode
Item

\item[{Default Value}] \leavevmode
\begin{DUlineblock}{0em}
\item[] None
\end{DUlineblock}

\end{description}\end{quote}

\item[{11.}] \leavevmode\begin{quote}\begin{description}
\item[{Name}] \leavevmode
Write

\item[{Description}] \leavevmode
Boolean to write files.

\item[{Data Access}] \leavevmode
Item

\item[{Default Value}] \leavevmode
\begin{DUlineblock}{0em}
\item[] False
\end{DUlineblock}

\end{description}\end{quote}

\item[{12.}] \leavevmode\begin{quote}\begin{description}
\item[{Name}] \leavevmode
Overwrite

\item[{Description}] \leavevmode
If True excising case will be overwritten. Default is set to True.

\item[{Data Access}] \leavevmode
Item

\item[{Default Value}] \leavevmode
\begin{DUlineblock}{0em}
\item[] True
\end{DUlineblock}

\end{description}\end{quote}

\item[{13.}] \leavevmode\begin{quote}\begin{description}
\item[{Name}] \leavevmode
Run

\item[{Description}] \leavevmode
\begin{DUlineblock}{0em}
\item[] Boolean to run analysis.
\item[] Analysis will be ran through SSH. Configure the connection with Livestock SSH.
\end{DUlineblock}

\item[{Data Access}] \leavevmode
Item

\item[{Default Value}] \leavevmode
\begin{DUlineblock}{0em}
\item[] False
\end{DUlineblock}

\end{description}\end{quote}

\end{description}\end{quote}

\item[{Outputs}] \leavevmode\begin{quote}\begin{description}
\item[{1.}] \leavevmode\begin{quote}\begin{description}
\item[{Name}] \leavevmode
readMe!

\item[{Description}] \leavevmode
\begin{DUlineblock}{0em}
\item[] In case of any errors, it will be shown here.
\end{DUlineblock}

\end{description}\end{quote}

\item[{2.}] \leavevmode\begin{quote}\begin{description}
\item[{Name}] \leavevmode
ResultPath

\item[{Description}] \leavevmode
\begin{DUlineblock}{0em}
\item[] Path to result files.
\end{DUlineblock}

\end{description}\end{quote}

\end{description}\end{quote}

\end{description}\end{quote}

\sphinxstylestrong{Livestock CMF Results}
\begin{quote}\begin{description}
\item[{Description}] \leavevmode
\begin{DUlineblock}{0em}
\item[] CMF Results
\end{DUlineblock}

\item[{Inputs}] \leavevmode\begin{quote}\begin{description}
\item[{1.}] \leavevmode\begin{quote}\begin{description}
\item[{Name}] \leavevmode
ResultFilePath

\item[{Description}] \leavevmode
Path to result file. Accepts output from Livestock Solve

\item[{Data Access}] \leavevmode
Item

\item[{Default Value}] \leavevmode
\begin{DUlineblock}{0em}
\item[] None
\end{DUlineblock}

\end{description}\end{quote}

\item[{2.}] \leavevmode\begin{quote}\begin{description}
\item[{Name}] \leavevmode
FetchResult

\item[{Description}] \leavevmode
\begin{DUlineblock}{0em}
\item[] Choose which result should be loaded:
\item[] 0 - Evapotranspiration
\item[] 1 - Surface water volume
\item[] 2 - Surface water flux
\item[] 3 - Heat flux
\item[] 4 - Aerodynamic resistance
\item[] 5 - Soil layer water flux
\item[] 6 - Soil layer potential
\item[] 7 - Soil layer theta
\item[] 8 - Soil layer volume
\item[] 9 - Soil layer wetness
\item[] Default is set to 0.
\end{DUlineblock}

\item[{Data Access}] \leavevmode
Item

\item[{Default Value}] \leavevmode
\begin{DUlineblock}{0em}
\item[] 0
\end{DUlineblock}

\end{description}\end{quote}

\item[{3.}] \leavevmode\begin{quote}\begin{description}
\item[{Name}] \leavevmode
SaveCSV

\item[{Description}] \leavevmode
Save the values as a csv file - Default is set to False.

\item[{Data Access}] \leavevmode
Item

\item[{Default Value}] \leavevmode
\begin{DUlineblock}{0em}
\item[] False
\end{DUlineblock}

\end{description}\end{quote}

\item[{4.}] \leavevmode\begin{quote}\begin{description}
\item[{Name}] \leavevmode
Run

\item[{Description}] \leavevmode
Run component.

\item[{Data Access}] \leavevmode
Item

\item[{Default Value}] \leavevmode
\begin{DUlineblock}{0em}
\item[] False
\end{DUlineblock}

\end{description}\end{quote}

\end{description}\end{quote}

\item[{Outputs}] \leavevmode\begin{quote}\begin{description}
\item[{1.}] \leavevmode\begin{quote}\begin{description}
\item[{Name}] \leavevmode
readMe!

\item[{Description}] \leavevmode
\begin{DUlineblock}{0em}
\item[] In case of any errors, it will be shown here.
\end{DUlineblock}

\end{description}\end{quote}

\item[{2.}] \leavevmode\begin{quote}\begin{description}
\item[{Name}] \leavevmode
Units

\item[{Description}] \leavevmode
\begin{DUlineblock}{0em}
\item[] Shows the units of the results.
\end{DUlineblock}

\end{description}\end{quote}

\item[{3.}] \leavevmode\begin{quote}\begin{description}
\item[{Name}] \leavevmode
Values

\item[{Description}] \leavevmode
\begin{DUlineblock}{0em}
\item[] List with chosen result values.
\end{DUlineblock}

\end{description}\end{quote}

\item[{4.}] \leavevmode\begin{quote}\begin{description}
\item[{Name}] \leavevmode
CSVPath

\item[{Description}] \leavevmode
\begin{DUlineblock}{0em}
\item[] Path to csv file.
\end{DUlineblock}

\end{description}\end{quote}

\end{description}\end{quote}

\end{description}\end{quote}

\sphinxstylestrong{Livestock CMF Outputs}
\begin{quote}\begin{description}
\item[{Description}] \leavevmode
CMF Outputs

\item[{Inputs}] \leavevmode\begin{quote}\begin{description}
\item[{1.}] \leavevmode\begin{quote}\begin{description}
\item[{Name}] \leavevmode
Evapotranspiration

\item[{Description}] \leavevmode
Cell evaporation - default is set to True.

\item[{Data Access}] \leavevmode
Item

\item[{Default Value}] \leavevmode
\begin{DUlineblock}{0em}
\item[] True
\end{DUlineblock}

\end{description}\end{quote}

\item[{2.}] \leavevmode\begin{quote}\begin{description}
\item[{Name}] \leavevmode
SurfaceWaterVolume

\item[{Description}] \leavevmode
Cell surface water - default is set to False.

\item[{Data Access}] \leavevmode
Item

\item[{Default Value}] \leavevmode
\begin{DUlineblock}{0em}
\item[] False
\end{DUlineblock}

\end{description}\end{quote}

\item[{3.}] \leavevmode\begin{quote}\begin{description}
\item[{Name}] \leavevmode
SurfaceWaterFlux

\item[{Description}] \leavevmode
Cell surface water flux - default is set to False.

\item[{Data Access}] \leavevmode
Item

\item[{Default Value}] \leavevmode
\begin{DUlineblock}{0em}
\item[] False
\end{DUlineblock}

\end{description}\end{quote}

\item[{4.}] \leavevmode\begin{quote}\begin{description}
\item[{Name}] \leavevmode
HeatFlux

\item[{Description}] \leavevmode
Cell surface heat flux - default is set to False.

\item[{Data Access}] \leavevmode
Item

\item[{Default Value}] \leavevmode
\begin{DUlineblock}{0em}
\item[] False
\end{DUlineblock}

\end{description}\end{quote}

\item[{5.}] \leavevmode\begin{quote}\begin{description}
\item[{Name}] \leavevmode
AerodynamicResistance

\item[{Description}] \leavevmode
Cell surface water - default is set to False.

\item[{Data Access}] \leavevmode
Item

\item[{Default Value}] \leavevmode
\begin{DUlineblock}{0em}
\item[] False
\end{DUlineblock}

\end{description}\end{quote}

\item[{6.}] \leavevmode\begin{quote}\begin{description}
\item[{Name}] \leavevmode
SurfaceWaterFlux

\item[{Description}] \leavevmode
Soil layer volumetric flux vectors - default is set to False.

\item[{Data Access}] \leavevmode
Item

\item[{Default Value}] \leavevmode
\begin{DUlineblock}{0em}
\item[] False
\end{DUlineblock}

\end{description}\end{quote}

\item[{7.}] \leavevmode\begin{quote}\begin{description}
\item[{Name}] \leavevmode
VolumetricFlux

\item[{Description}] \leavevmode
Soil layer volumetric flux vectors - default is set to False.

\item[{Data Access}] \leavevmode
Item

\item[{Default Value}] \leavevmode
\begin{DUlineblock}{0em}
\item[] False
\end{DUlineblock}

\end{description}\end{quote}

\item[{8.}] \leavevmode\begin{quote}\begin{description}
\item[{Name}] \leavevmode
Potential

\item[{Description}] \leavevmode
Soil layer total potential (Psi$_{\text{tot}}$= Psi$_{\text{M}}$+ Psi$_{\text{G}}$- default is set to False.

\item[{Data Access}] \leavevmode
Item

\item[{Default Value}] \leavevmode
\begin{DUlineblock}{0em}
\item[] False
\end{DUlineblock}

\end{description}\end{quote}

\item[{9.}] \leavevmode\begin{quote}\begin{description}
\item[{Name}] \leavevmode
Theta

\item[{Description}] \leavevmode
Soil layer volumetric water content of the layer - default is set to False.

\item[{Data Access}] \leavevmode
Item

\item[{Default Value}] \leavevmode
\begin{DUlineblock}{0em}
\item[] False
\end{DUlineblock}

\end{description}\end{quote}

\item[{10.}] \leavevmode\begin{quote}\begin{description}
\item[{Name}] \leavevmode
Volume

\item[{Description}] \leavevmode
Soil layer volume of water in the layer - default is set to True.

\item[{Data Access}] \leavevmode
Item

\item[{Default Value}] \leavevmode
\begin{DUlineblock}{0em}
\item[] True
\end{DUlineblock}

\end{description}\end{quote}

\item[{10.}] \leavevmode\begin{quote}\begin{description}
\item[{Name}] \leavevmode
Wetness

\item[{Description}] \leavevmode
Soil layer wetness of the soil (V$_{\text{volume}}$/V$_{\text{pores}}$) - default is set to False.

\item[{Data Access}] \leavevmode
Item

\item[{Default Value}] \leavevmode
\begin{DUlineblock}{0em}
\item[] False
\end{DUlineblock}

\end{description}\end{quote}

\end{description}\end{quote}

\item[{Outputs}] \leavevmode\begin{quote}\begin{description}
\item[{1.}] \leavevmode\begin{quote}\begin{description}
\item[{Name}] \leavevmode
readMe!

\item[{Description}] \leavevmode
\begin{DUlineblock}{0em}
\item[] In case of any errors, it will be shown here.
\end{DUlineblock}

\end{description}\end{quote}

\item[{2.}] \leavevmode\begin{quote}\begin{description}
\item[{Name}] \leavevmode
ChosenOutputs

\item[{Description}] \leavevmode
\begin{DUlineblock}{0em}
\item[] Shows the chosen outputs.
\end{DUlineblock}

\end{description}\end{quote}

\item[{3.}] \leavevmode\begin{quote}\begin{description}
\item[{Name}] \leavevmode
Outputs

\item[{Description}] \leavevmode
\begin{DUlineblock}{0em}
\item[] Livestock Output Data.
\end{DUlineblock}

\end{description}\end{quote}

\end{description}\end{quote}

\end{description}\end{quote}

\sphinxstylestrong{Livestock CMF Boundary Condition}
\begin{quote}\begin{description}
\item[{Description}] \leavevmode
CMF Boundary connection

\item[{Inputs}] \leavevmode\begin{quote}\begin{description}
\item[{1.}] \leavevmode\begin{quote}\begin{description}
\item[{Name}] \leavevmode
InletOrOutlet

\item[{Description}] \leavevmode
0 is inlet. 1 is outlet - default is set to 0

\item[{Data Access}] \leavevmode
Item

\item[{Default Value}] \leavevmode
\begin{DUlineblock}{0em}
\item[] 0
\end{DUlineblock}

\end{description}\end{quote}

\item[{2.}] \leavevmode\begin{quote}\begin{description}
\item[{Name}] \leavevmode
ConnectedCell

\item[{Description}] \leavevmode
Cell to connect to. Default is set to first cell.

\item[{Data Access}] \leavevmode
Item

\item[{Default Value}] \leavevmode
\begin{DUlineblock}{0em}
\item[] 0
\end{DUlineblock}

\end{description}\end{quote}

\item[{3.}] \leavevmode\begin{quote}\begin{description}
\item[{Name}] \leavevmode
ConnectedLayer

\item[{Description}] \leavevmode
Layer of cell to connect to. 0 is surface water. 1 is first layer of cell and so on.
Default is set to 0 - surface water.

\item[{Data Access}] \leavevmode
Item

\item[{Default Value}] \leavevmode
\begin{DUlineblock}{0em}
\item[] 0
\end{DUlineblock}

\end{description}\end{quote}

\item[{4.}] \leavevmode\begin{quote}\begin{description}
\item[{Name}] \leavevmode
InletFlux

\item[{Description}] \leavevmode
If inlet, then set flux in m3/day.

\item[{Data Access}] \leavevmode
List

\item[{Default Value}] \leavevmode
\begin{DUlineblock}{0em}
\item[] False
\end{DUlineblock}

\end{description}\end{quote}

\item[{5.}] \leavevmode\begin{quote}\begin{description}
\item[{Name}] \leavevmode
FlowWidth

\item[{Description}] \leavevmode
Width of the connection from cell to outlet in meters.

\item[{Data Access}] \leavevmode
Item

\item[{Default Value}] \leavevmode
\begin{DUlineblock}{0em}
\item[] None
\end{DUlineblock}

\end{description}\end{quote}

\item[{6.}] \leavevmode\begin{quote}\begin{description}
\item[{Name}] \leavevmode
OutletLocation

\item[{Description}] \leavevmode
Location of the outlet in x, y and z coordinates.

\item[{Data Access}] \leavevmode
List

\item[{Default Value}] \leavevmode
\begin{DUlineblock}{0em}
\item[] None
\end{DUlineblock}

\end{description}\end{quote}

\end{description}\end{quote}

\item[{Outputs}] \leavevmode\begin{quote}\begin{description}
\item[{1.}] \leavevmode\begin{quote}\begin{description}
\item[{Name}] \leavevmode
readMe!

\item[{Description}] \leavevmode
\begin{DUlineblock}{0em}
\item[] In case of any errors, it will be shown here.
\end{DUlineblock}

\end{description}\end{quote}

\item[{2.}] \leavevmode\begin{quote}\begin{description}
\item[{Name}] \leavevmode
BoundaryCondition

\item[{Description}] \leavevmode
\begin{DUlineblock}{0em}
\item[] Livestock Boundary Conditions.
\end{DUlineblock}

\end{description}\end{quote}

\end{description}\end{quote}

\end{description}\end{quote}

\sphinxstylestrong{Livestock CMF Solver Settings}
\begin{quote}\begin{description}
\item[{Description}] \leavevmode
Sets the solver settings for CMF Solve

\item[{Inputs}] \leavevmode\begin{quote}\begin{description}
\item[{1.}] \leavevmode\begin{quote}\begin{description}
\item[{Name}] \leavevmode
AnalysisLength

\item[{Description}] \leavevmode
Number of time steps to be taken - Default is 24

\item[{Data Access}] \leavevmode
Item

\item[{Default Value}] \leavevmode
\begin{DUlineblock}{0em}
\item[] 24
\end{DUlineblock}

\end{description}\end{quote}

\item[{2.}] \leavevmode\begin{quote}\begin{description}
\item[{Name}] \leavevmode
TimeStep

\item[{Description}] \leavevmode
Size of each time step in hours - e.g. 1/60 equals time steps of 1 min and 24 is a time step
of one day. Default is 1 hour.

\item[{Data Access}] \leavevmode
Item

\item[{Default Value}] \leavevmode
\begin{DUlineblock}{0em}
\item[] 1
\end{DUlineblock}

\end{description}\end{quote}

\item[{3.}] \leavevmode\begin{quote}\begin{description}
\item[{Name}] \leavevmode
SolverTolerance

\item[{Description}] \leavevmode
Solver tolerance - Default is 1e-8

\item[{Data Access}] \leavevmode
Item

\item[{Default Value}] \leavevmode
\begin{DUlineblock}{0em}
\item[] 10**-8
\end{DUlineblock}

\end{description}\end{quote}

\item[{4.}] \leavevmode\begin{quote}\begin{description}
\item[{Name}] \leavevmode
Verbosity

\item[{Description}] \leavevmode
\begin{DUlineblock}{0em}
\item[] Sets the verbosity of the print statement during runtime - Default is 1.
\item[] 0 - Prints only at start and end of simulation.
\item[] 1 - Prints at every time step.
\end{DUlineblock}

\item[{Data Access}] \leavevmode
Item

\item[{Default Value}] \leavevmode
\begin{DUlineblock}{0em}
\item[] 1
\end{DUlineblock}

\end{description}\end{quote}

\end{description}\end{quote}

\item[{Outputs}] \leavevmode\begin{quote}\begin{description}
\item[{1.}] \leavevmode\begin{quote}\begin{description}
\item[{Name}] \leavevmode
readMe!

\item[{Description}] \leavevmode
\begin{DUlineblock}{0em}
\item[] In case of any errors, it will be shown here.
\end{DUlineblock}

\end{description}\end{quote}

\item[{2.}] \leavevmode\begin{quote}\begin{description}
\item[{Name}] \leavevmode
SolverSettings

\item[{Description}] \leavevmode
\begin{DUlineblock}{0em}
\item[] Livestock Solver Settings.
\end{DUlineblock}

\end{description}\end{quote}

\end{description}\end{quote}

\end{description}\end{quote}

\sphinxstylestrong{Livestock CMF Surface Flux Result}
\begin{quote}\begin{description}
\item[{Description}] \leavevmode
Extract the surface flux for a mesh.

\item[{Inputs}] \leavevmode\begin{quote}\begin{description}
\item[{1.}] \leavevmode\begin{quote}\begin{description}
\item[{Name}] \leavevmode
ResultFilePath

\item[{Description}] \leavevmode
Path to result file. Accepts output from Livestock Solve

\item[{Data Access}] \leavevmode
Item

\item[{Default Value}] \leavevmode
\begin{DUlineblock}{0em}
\item[] None
\end{DUlineblock}

\end{description}\end{quote}

\item[{2.}] \leavevmode\begin{quote}\begin{description}
\item[{Name}] \leavevmode
Mesh

\item[{Description}] \leavevmode
Mesh of the case

\item[{Data Access}] \leavevmode
Item

\item[{Default Value}] \leavevmode
\begin{DUlineblock}{0em}
\item[] None
\end{DUlineblock}

\end{description}\end{quote}

\item[{3.}] \leavevmode\begin{quote}\begin{description}
\item[{Name}] \leavevmode
IncludeRunOff

\item[{Description}] \leavevmode
Include surface run-off into the surface flux vector? Default is set to True.

\item[{Data Access}] \leavevmode
Item

\item[{Default Value}] \leavevmode
\begin{DUlineblock}{0em}
\item[] True
\end{DUlineblock}

\end{description}\end{quote}

\item[{4.}] \leavevmode\begin{quote}\begin{description}
\item[{Name}] \leavevmode
IncludeRain

\item[{Description}] \leavevmode
Include rain into the surface flux vector? Default is False.

\item[{Data Access}] \leavevmode
Item

\item[{Default Value}] \leavevmode
\begin{DUlineblock}{0em}
\item[] False
\end{DUlineblock}

\end{description}\end{quote}

\item[{5.}] \leavevmode\begin{quote}\begin{description}
\item[{Name}] \leavevmode
IncludeEvapotranspiration

\item[{Description}] \leavevmode
Include evapotranspiration into the surface flux vector? Default is set to False.

\item[{Data Access}] \leavevmode
Item

\item[{Default Value}] \leavevmode
\begin{DUlineblock}{0em}
\item[] False
\end{DUlineblock}

\end{description}\end{quote}

\item[{6.}] \leavevmode\begin{quote}\begin{description}
\item[{Name}] \leavevmode
IncludeInfiltration

\item[{Description}] \leavevmode
Include infiltration into the surface flux vector? Default is False.

\item[{Data Access}] \leavevmode
Item

\item[{Default Value}] \leavevmode
\begin{DUlineblock}{0em}
\item[] False
\end{DUlineblock}

\end{description}\end{quote}

\item[{7.}] \leavevmode\begin{quote}\begin{description}
\item[{Name}] \leavevmode
SaveResult

\item[{Description}] \leavevmode
Save the values as a text file - Default is set to False.

\item[{Data Access}] \leavevmode
Item

\item[{Default Value}] \leavevmode
\begin{DUlineblock}{0em}
\item[] False
\end{DUlineblock}

\end{description}\end{quote}

\item[{8.}] \leavevmode\begin{quote}\begin{description}
\item[{Name}] \leavevmode
Run

\item[{Description}] \leavevmode
Run component. Default is False.

\item[{Data Access}] \leavevmode
Item

\item[{Default Value}] \leavevmode
\begin{DUlineblock}{0em}
\item[] False
\end{DUlineblock}

\end{description}\end{quote}

\end{description}\end{quote}

\item[{Outputs}] \leavevmode\begin{quote}\begin{description}
\item[{1.}] \leavevmode\begin{quote}\begin{description}
\item[{Name}] \leavevmode
readMe!

\item[{Description}] \leavevmode
\begin{DUlineblock}{0em}
\item[] In case of any errors, it will be shown here.
\end{DUlineblock}

\end{description}\end{quote}

\item[{2.}] \leavevmode\begin{quote}\begin{description}
\item[{Name}] \leavevmode
Unit

\item[{Description}] \leavevmode
\begin{DUlineblock}{0em}
\item[] Shows the units of the results.
\end{DUlineblock}

\end{description}\end{quote}

\item[{3.}] \leavevmode\begin{quote}\begin{description}
\item[{Name}] \leavevmode
SurfaceFluxVectors

\item[{Description}] \leavevmode
\begin{DUlineblock}{0em}
\item[] Tree with the surface flux vectors.
\end{DUlineblock}

\end{description}\end{quote}

\item[{4.}] \leavevmode\begin{quote}\begin{description}
\item[{Name}] \leavevmode
CSVPath

\item[{Description}] \leavevmode
\begin{DUlineblock}{0em}
\item[] Path to csv file.
\end{DUlineblock}

\end{description}\end{quote}

\end{description}\end{quote}

\end{description}\end{quote}


\section{4 \textbar{} Comfort}
\label{\detokenize{components:comfort}}
\sphinxstylestrong{Livestock New Air Conditions}
\begin{quote}\begin{description}
\item[{Description}] \leavevmode
Computes new air temperature and relative humidity

\item[{Inputs}] \leavevmode\begin{quote}\begin{description}
\item[{1.}] \leavevmode\begin{quote}\begin{description}
\item[{Name}] \leavevmode
Mesh

\item[{Description}] \leavevmode
Ground Mesh

\item[{Data Access}] \leavevmode
Item

\item[{Default Value}] \leavevmode
\begin{DUlineblock}{0em}
\item[] None
\end{DUlineblock}

\end{description}\end{quote}

\item[{2.}] \leavevmode\begin{quote}\begin{description}
\item[{Name}] \leavevmode
Evapotranspiration

\item[{Description}] \leavevmode
Evapotranspiration in m$^{\text{3}}$/day.
Each tree branch should represent one time unit, with all the cell values to that time.

\item[{Data Access}] \leavevmode
Tree

\item[{Default Value}] \leavevmode
\begin{DUlineblock}{0em}
\item[] None
\end{DUlineblock}

\end{description}\end{quote}

\item[{3.}] \leavevmode\begin{quote}\begin{description}
\item[{Name}] \leavevmode
HeatFlux

\item[{Description}] \leavevmode
HeatFlux in MJ/m$^{\text{2}}$day.
Each tree branch should represent one time unit, with all the cell values to that time.

\item[{Data Access}] \leavevmode
Tree

\item[{Default Value}] \leavevmode
\begin{DUlineblock}{0em}
\item[] None
\end{DUlineblock}

\end{description}\end{quote}

\item[{4.}] \leavevmode\begin{quote}\begin{description}
\item[{Name}] \leavevmode
AirTemperature

\item[{Description}] \leavevmode
Air temperature in C

\item[{Data Access}] \leavevmode
List

\item[{Default Value}] \leavevmode
\begin{DUlineblock}{0em}
\item[] None
\end{DUlineblock}

\end{description}\end{quote}

\item[{5.}] \leavevmode\begin{quote}\begin{description}
\item[{Name}] \leavevmode
AirRelativeHumidity

\item[{Description}] \leavevmode
Relative Humidity in -

\item[{Data Access}] \leavevmode
List

\item[{Default Value}] \leavevmode
\begin{DUlineblock}{0em}
\item[] None
\end{DUlineblock}

\end{description}\end{quote}

\item[{6.}] \leavevmode\begin{quote}\begin{description}
\item[{Name}] \leavevmode
AirBoundaryHeight

\item[{Description}] \leavevmode
Top of the air column in m. Default is set to 10m.

\item[{Data Access}] \leavevmode
Item

\item[{Default Value}] \leavevmode
\begin{DUlineblock}{0em}
\item[] 10
\end{DUlineblock}

\end{description}\end{quote}

\item[{7.}] \leavevmode\begin{quote}\begin{description}
\item[{Name}] \leavevmode
InvestigationHeight

\item[{Description}] \leavevmode
Height at which the new air temperature and relative humidity should be calculated.
Default is set to 1.1m.

\item[{Data Access}] \leavevmode
Item

\item[{Default Value}] \leavevmode
\begin{DUlineblock}{0em}
\item[] 1.1
\end{DUlineblock}

\end{description}\end{quote}

\item[{8.}] \leavevmode\begin{quote}\begin{description}
\item[{Name}] \leavevmode
CPUs

\item[{Description}] \leavevmode
Number of CPUs to perform the computations on. Default is set to 2

\item[{Data Access}] \leavevmode
Item

\item[{Default Value}] \leavevmode
\begin{DUlineblock}{0em}
\item[] 2
\end{DUlineblock}

\end{description}\end{quote}

\item[{9.}] \leavevmode\begin{quote}\begin{description}
\item[{Name}] \leavevmode
ThroughSSH

\item[{Description}] \leavevmode
If the computation should be run through SSH. Default is set to False

\item[{Data Access}] \leavevmode
Item

\item[{Default Value}] \leavevmode
\begin{DUlineblock}{0em}
\item[] False
\end{DUlineblock}

\end{description}\end{quote}

\item[{10.}] \leavevmode\begin{quote}\begin{description}
\item[{Name}] \leavevmode
Run

\item[{Description}] \leavevmode
Run the component

\item[{Data Access}] \leavevmode
Item

\item[{Default Value}] \leavevmode
\begin{DUlineblock}{0em}
\item[] False
\end{DUlineblock}

\end{description}\end{quote}

\end{description}\end{quote}

\item[{Outputs}] \leavevmode\begin{quote}\begin{description}
\item[{1.}] \leavevmode\begin{quote}\begin{description}
\item[{Name}] \leavevmode
readMe!

\item[{Description}] \leavevmode
\begin{DUlineblock}{0em}
\item[] In case of any errors, it will be shown here.
\end{DUlineblock}

\end{description}\end{quote}

\item[{2.}] \leavevmode\begin{quote}\begin{description}
\item[{Name}] \leavevmode
NewTemperature

\item[{Description}] \leavevmode
\begin{DUlineblock}{0em}
\item[] New temperature in C.
\end{DUlineblock}

\end{description}\end{quote}

\item[{3.}] \leavevmode\begin{quote}\begin{description}
\item[{Name}] \leavevmode
NewRelativeHumidity

\item[{Description}] \leavevmode
\begin{DUlineblock}{0em}
\item[] New relative humidity in -.
\end{DUlineblock}

\end{description}\end{quote}

\end{description}\end{quote}

\end{description}\end{quote}

\sphinxstylestrong{Livestock Adaptive Clothing}
\begin{quote}\begin{description}
\item[{Description}] \leavevmode
\begin{DUlineblock}{0em}
\item[] Computes the clothing isolation in clo for a given outdoor temperature.
\item[] Source: Havenith et al. - 2012 - “The UTCI-clothing model”
\end{DUlineblock}

\item[{Inputs}] \leavevmode\begin{quote}\begin{description}
\item[{1.}] \leavevmode\begin{quote}\begin{description}
\item[{Name}] \leavevmode
Temperature

\item[{Description}] \leavevmode
Temperature in C

\item[{Data Access}] \leavevmode
List

\item[{Default Value}] \leavevmode
\begin{DUlineblock}{0em}
\item[] None
\end{DUlineblock}

\end{description}\end{quote}

\end{description}\end{quote}

\item[{Outputs}] \leavevmode\begin{quote}\begin{description}
\item[{1.}] \leavevmode\begin{quote}\begin{description}
\item[{Name}] \leavevmode
readMe!

\item[{Description}] \leavevmode
\begin{DUlineblock}{0em}
\item[] In case of any errors, it will be shown here.
\end{DUlineblock}

\end{description}\end{quote}

\item[{2.}] \leavevmode\begin{quote}\begin{description}
\item[{Name}] \leavevmode
ClothingValue

\item[{Description}] \leavevmode
\begin{DUlineblock}{0em}
\item[] Calculated clothing value in clo.
\end{DUlineblock}

\end{description}\end{quote}

\end{description}\end{quote}

\end{description}\end{quote}


\chapter{Livestock Grasshopper Component Classes}
\label{\detokenize{component classes::doc}}\label{\detokenize{component classes:livestock-grasshopper-component-classes}}

\section{SuperClass}
\label{\detokenize{component classes:module-livestock.components.component}}\label{\detokenize{component classes:superclass}}\index{livestock.components.component (module)}

\section{0 \textbar{} Miscellaneous}
\label{\detokenize{component classes:module-livestock.components.misc}}\label{\detokenize{component classes:miscellaneous}}\index{livestock.components.misc (module)}

\section{1 \textbar{} Geometry}
\label{\detokenize{component classes:geometry}}\label{\detokenize{component classes:module-livestock.components.geometry}}\index{livestock.components.geometry (module)}

\section{3 \textbar{} CMF}
\label{\detokenize{component classes:module-livestock.components.comp_cmf}}\label{\detokenize{component classes:cmf}}\index{livestock.components.comp\_cmf (module)}

\section{4 \textbar{} Comfort}
\label{\detokenize{component classes:comfort}}\label{\detokenize{component classes:module-livestock.components.comfort}}\index{livestock.components.comfort (module)}

\chapter{Livestock Grasshopper Lib}
\label{\detokenize{lib:livestock-grasshopper-lib}}\label{\detokenize{lib::doc}}

\section{Geometry}
\label{\detokenize{lib:geometry}}\label{\detokenize{lib:module-livestock.lib.geometry}}\index{livestock.lib.geometry (module)}

\section{Miscellaneous}
\label{\detokenize{lib:module-livestock.lib.misc}}\label{\detokenize{lib:miscellaneous}}\index{livestock.lib.misc (module)}

\section{SSH}
\label{\detokenize{lib:module-livestock.lib.ssh}}\label{\detokenize{lib:ssh}}\index{livestock.lib.ssh (module)}

\section{Templates}
\label{\detokenize{lib:module-livestock.lib.templates}}\label{\detokenize{lib:templates}}\index{livestock.lib.templates (module)}\index{pick\_template() (in module livestock.lib.templates)}

\begin{fulllineitems}
\phantomsection\label{\detokenize{lib:livestock.lib.templates.pick_template}}\pysiglinewithargsret{\sphinxcode{livestock.lib.templates.}\sphinxbfcode{pick\_template}}{\emph{template\_name}, \emph{path}}{}
Writes a template given a template name and path to write it to.
:param template\_name: Template name.
:param path: Path to save it to.

\end{fulllineitems}

\index{drain\_mesh\_template() (in module livestock.lib.templates)}

\begin{fulllineitems}
\phantomsection\label{\detokenize{lib:livestock.lib.templates.drain_mesh_template}}\pysiglinewithargsret{\sphinxcode{livestock.lib.templates.}\sphinxbfcode{drain\_mesh\_template}}{\emph{path}}{}
Writes the template for the drain mesh function.
:param path: Path to write it to.

\end{fulllineitems}

\index{ssh\_template() (in module livestock.lib.templates)}

\begin{fulllineitems}
\phantomsection\label{\detokenize{lib:livestock.lib.templates.ssh_template}}\pysiglinewithargsret{\sphinxcode{livestock.lib.templates.}\sphinxbfcode{ssh\_template}}{\emph{path}}{}
Writes the ssh template.
:param path: Path to write it to.

\end{fulllineitems}

\index{cmf\_template() (in module livestock.lib.templates)}

\begin{fulllineitems}
\phantomsection\label{\detokenize{lib:livestock.lib.templates.cmf_template}}\pysiglinewithargsret{\sphinxcode{livestock.lib.templates.}\sphinxbfcode{cmf\_template}}{\emph{path}}{}
Writes the CMF template.
:param path: Path to write it to.

\end{fulllineitems}

\index{process\_cmf\_results() (in module livestock.lib.templates)}

\begin{fulllineitems}
\phantomsection\label{\detokenize{lib:livestock.lib.templates.process_cmf_results}}\pysiglinewithargsret{\sphinxcode{livestock.lib.templates.}\sphinxbfcode{process\_cmf\_results}}{\emph{path}}{}
Writes the CMF result lookup template.
:param path: Path to write it to.

\end{fulllineitems}

\index{process\_cmf\_surface\_results() (in module livestock.lib.templates)}

\begin{fulllineitems}
\phantomsection\label{\detokenize{lib:livestock.lib.templates.process_cmf_surface_results}}\pysiglinewithargsret{\sphinxcode{livestock.lib.templates.}\sphinxbfcode{process\_cmf\_surface\_results}}{\emph{path}}{}
Writes the CMF surface result template.
:param path: Path to write it to.

\end{fulllineitems}

\index{new\_air\_conditions() (in module livestock.lib.templates)}

\begin{fulllineitems}
\phantomsection\label{\detokenize{lib:livestock.lib.templates.new_air_conditions}}\pysiglinewithargsret{\sphinxcode{livestock.lib.templates.}\sphinxbfcode{new\_air\_conditions}}{\emph{path}}{}
Writes the new air condition template.
:param path: Path to write it to.

\end{fulllineitems}



\chapter{Indices and tables}
\label{\detokenize{index:indices-and-tables}}\begin{itemize}
\item {} 
\DUrole{xref,std,std-ref}{genindex}

\item {} 
\DUrole{xref,std,std-ref}{modindex}

\item {} 
\DUrole{xref,std,std-ref}{search}

\end{itemize}


\renewcommand{\indexname}{Python Module Index}
\begin{sphinxtheindex}
\def\bigletter#1{{\Large\sffamily#1}\nopagebreak\vspace{1mm}}
\bigletter{l}
\item {\sphinxstyleindexentry{livestock.components.comfort}}\sphinxstyleindexpageref{component classes:\detokenize{module-livestock.components.comfort}}
\item {\sphinxstyleindexentry{livestock.components.comp\_cmf}}\sphinxstyleindexpageref{component classes:\detokenize{module-livestock.components.comp_cmf}}
\item {\sphinxstyleindexentry{livestock.components.component}}\sphinxstyleindexpageref{component classes:\detokenize{module-livestock.components.component}}
\item {\sphinxstyleindexentry{livestock.components.geometry}}\sphinxstyleindexpageref{component classes:\detokenize{module-livestock.components.geometry}}
\item {\sphinxstyleindexentry{livestock.components.misc}}\sphinxstyleindexpageref{component classes:\detokenize{module-livestock.components.misc}}
\item {\sphinxstyleindexentry{livestock.lib.geometry}}\sphinxstyleindexpageref{lib:\detokenize{module-livestock.lib.geometry}}
\item {\sphinxstyleindexentry{livestock.lib.misc}}\sphinxstyleindexpageref{lib:\detokenize{module-livestock.lib.misc}}
\item {\sphinxstyleindexentry{livestock.lib.ssh}}\sphinxstyleindexpageref{lib:\detokenize{module-livestock.lib.ssh}}
\item {\sphinxstyleindexentry{livestock.lib.templates}}\sphinxstyleindexpageref{lib:\detokenize{module-livestock.lib.templates}}
\end{sphinxtheindex}

\renewcommand{\indexname}{Index}
\printindex
\end{document}